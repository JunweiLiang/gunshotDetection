%% ========================================================================
%%  JANUS-SR   Janus Speech Recognition Toolkit
%%             ------------------------------------------------------------
%%             Object: Basic Janus UI
%%             ------------------------------------------------------------
%%
%%  Author  :  Florian Metze & many others
%%  Module  :  basicJanus.tex
%%  Date    :  $Id: basicJanus.tex 2390 2003-08-14 11:20:32Z fuegen $
%%
%%  Remarks :  
%%
%% ========================================================================
%%
%%  $Log$
%%  Revision 1.2  2003/08/14 11:18:44  fuegen
%%  Merged changes on branch jtk-01-01-15-fms (jaguar -> ibis-013)
%%
%%  Revision 1.1.2.2  2002/08/26 16:40:37  metze
%%  Re-organization of modules
%%
%%  Revision 1.1.2.1  2002/08/01 13:42:20  metze
%%  Fixes for clean documentation.
%%
%% ========================================================================

\section{Janus Objects}

JANUS was designed to be programmable. The programming language is
Tcl/Tk, expanded by some object classes and their methods. Object
classes are things like dictionaries, codebooks, but also the decoder
itself is an object class. Every object class has its methods
(operations that can be done with objects of that class). Objects can
have subobjects and can be hierarchically organized. The object oriented
programming paradign allows, at least in principle, to plug in and out
objects as one wishes. Simply change the dictionary by assigning a new
one, copy codebooks as easily as "cb1 := cb2", add distribution
accumulators as easily as "ds1.accu += ds2.accu", etc.

\subsection*{Create JANUS Objects}

Objects are meant to hold data but also provide methods to manipulate
that data. To define an object you have to specify its *type*. The
convention is that type names start with capital letters and objects
with small letters. You can define as many objects of one type as you
like. To see what types exist just type (one of the few) JANUS command
'types'.\footnote{By the way '\%' is the prompt of the JANUS shell.
 You can also use any Tcl and Tk commands.}

\begin{verbatim}
% types
FlatFwd ModelArray DurationSet Word Cbcfg SampleSet DVector PTree HMM 
Vocab FMatrix CodebookMapItem PTreeSet MLNorm Dscfg PathItemList 
CodebookAccu HypoList DBaseIdx SampleSetClass SenoneTag Feature 
PhonesSet DCovMatrix Phone FBMatrix Phones DistribAccu IMatrix 
TopoSet SVector XWModel IArray DMatrix FVector StateGraph FCovMatrix 
Duration PTreeNode LatNode Hypo Senone TreeFwd LDA Topo MLNormClass 
Codebook Tags LDAClass FeatureSet Tree PhoneGraph CodebookMap Path 
Search AModelSet RewriteSet
\end{verbatim}

The list you get here depends on the version and compile options. You
create an object when you enter a *type name* followed by the *name of
the new object*. Some types require additional arguments, like
subobjects. As an example we define an object (let's call it 'ps') of
type PhonesSet. You can get a list of all objects you have defined with
the command 'objects'. One object name can only be used once (also for
different types) but you can 'destroy' objects. 'destroy' is a standard
method of every object (s.b.).

\begin{verbatim}
% PhonesSet ps
ps
% PhonesSet ps
WARNING itf.c(0287)   Object ps already exists.
% PhonesSet ps2 
ps2
% objects
ps ps2
\end{verbatim}

summary: JANUS commands *types * list all object types
*objects * list all objects defined by user

\subsection*{Standard Methods}

As soon as you have defined an object you can use its *name* followed by
a *method* and arguments. The different object types have their own
methods of course but at least a few are standard methods that exist for
every object. These are

\begin{description}
\item[type] gives the type of the object
\item[puts] print contents of the object
\item[configure] configure the object
\item[:] allow access to list element
\item[.] allow access to subobjects
\item[destroy] destroy object
\end{description}

To find out what other methods exist we enter eather the *type name*
without any object name or the object name follwed by '-help'. To get
more information about a specific method we enter the object name, the
method and '-help'.

\begin{verbatim}
% ps -help

DESCRIPTION

A 'PhonesSet' object is a set of 'Phones' objects.

METHODS

puts               displays the contents of a set of phone-sets
add                add new phone-set to a set of phones-set
delete             delete phone-set(s) from a set of phone-sets
read               read a set of phone-sets from a file
write              write a set of phone-sets into a file
index              return index of named phone-set(s)
name               return the name of indexed phone-set(s)

% ps add -help
Options of 'add' are:
 < name>    name of list (string:"NULL")
 < phone*>  list of phones
% ps add VOWEL A E I O U
\end{verbatim}

We just added the element 'VOWEL' to the PhonesSet object ps. To see the
contents of the object we can use the method 'puts' or just the object
name which is the same in most cases.

\begin{verbatim}
% ps puts 
VOWEL
% ps
VOWEL
\end{verbatim}

\subsection*{Access to Elements and Subobjects}

The standard methods ':' and '.' allow access to elements and subobjects
respectively. *Elements* of an objects have the same kind of structure
like the words of a dictionary or phone groups (like the 'VOWEL') in the
PhonesSet. Nevertheless they can also be objects and most time they are.
*Subobjects* are more unique, like the Phones of a dictionary as we will
see immedeately. For these two methods and only for them you can omit
the spaces between the object and also between the single argument which
is the name of the element or the subobject. If you don't give a name
you will obtain a list of the choices.\footnote{In case of ':' you again 
get a list of the elements like with 'puts' or just the object name with 
no method.}

\begin{verbatim}
% ps:
VOWEL
% ps:VOWEL
A E I O U
% ps type
PhonesSet
% ps:VOWEL type
Phones
\end{verbatim}

Let's assume we have also defined a phone group 'PHONES' in the
PhonesSet 'ps' that contains all the Phones of a dictionary. Then we can
create a dictionary object that needs the name of a *Phones object* and
of a *Tags object* as arguments. Both have to be created before.

\begin{verbatim}
% Tags ts
ts
% Dictionary d ps:PHONES ts
d
% d add DOG "D O G"
% d add DUCK "D U CK"
% d.
phones tags item(0..1) list
% d:
DOG DUCK
\end{verbatim}

With 'd.phones' for example you have access to the object 'ps:PHONES'.
Although there is a method for PhonesSet to delete elements like
'PHONES' you will get an error if you try that because it was locked as
you defined the dictionary. This prevents objects from being deleted
while they are used by other objects.

\subsection*{Configuration}

Sometimes it might be necessary to configure *objects* or *object
types*. You can get all configure items, get a specific one or set one
or more items. In the latter case only if they are writable.

\begin{verbatim}
% ps configure
{-useN 1} {-commentChar {;}} {-itemN 2} {-blkSize 20} 
% ps configure -commentChar
{;}
% ps configure -commentChar #
\end{verbatim}

Note: The old comment sign ';' was protected with curly braces {}
because it is the command separator in Tcl.

Can you explain the following line and its return value.

\begin{verbatim}
% ps configure -commentChar ;
#
\end{verbatim}

%%% Local Variables: 
%%% mode: latex
%%% TeX-master: t
%%% End: 
