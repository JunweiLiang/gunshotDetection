%% ========================================================================
%%  JANUS-SR   Janus Speech Recognition Toolkit
%%             ------------------------------------------------------------
%%             Object: Description of a context free grammar
%%             ------------------------------------------------------------
%%
%%  Author  :  Christian Fuegen & many others
%%  Module  :  cfg.tex
%%  Date    :  $Id: cfg.tex 2390 2003-08-14 11:20:32Z fuegen $
%%
%%  Remarks :  
%%
%% ========================================================================
%%
%%   $Log$
%%   Revision 1.2  2003/08/14 11:18:57  fuegen
%%   Merged changes on branch jtk-01-01-15-fms (jaguar -> ibis-013)
%%
%%   Revision 1.1.2.2  2003/04/30 15:44:01  metze
%%   Changes before the new repository
%%
%%   Revision 1.1.2.1  2002/11/19 08:38:38  fuegen
%%   initial version
%%
%%
%% ========================================================================

\section{\Jlabel{file}{ContextFreeGrammars}}

This section describes our internal context free grammar format,
called SOUP-Format. We are usually using semantic instead of syntactic
context free grammars. They are read by \Jref{module}{CFGSet}. A not
completely specified example looks like:

\begin{verbatim}
# ======================================================================
# example grammar
# ======================================================================

# ----------------------------------------
# request path description
#       how do i
#       i want to find the way
#       can you take me
# ----------------------------------------
s[request-path-description]
        ( *PLEASE [_NT_how-to-go]    [obj_desc]     *PLEASE )
        ( *PLEASE [_NT_find-the-way] [obj_desc]     *PLEASE )
        ( *PLEASE [_NT_take-me]      [obj_desc]     *PLEASE )
        ( *PLEASE [_NT_how-about]    [obj_desc]     *PLEASE )
        ( *PLEASE [_NT_how-to-find]  [_NT_obj_desc] *PLEASE )

[_NT_how-to-go]
        ( how do i                           GO )
        ( [_NT_can-you-show|tell] *me how to GO )
        ( i WANT                          to GO )
        ( i NEED                          to GO )

[_NT_can-you-show|tell]
        ( *CAN_YOU SHOW )
        ( *CAN_YOU TELL )


[obj_desc]
        (                     to [_NT_obj_desc] )
        ( from [_NT_obj_desc] to [_NT_obj_desc] )

[_NT_obj_desc]
        ( *the          [objnm] )
        (  the *NEAREST [objcl] )
        (  A            [objcl] *NEARBY )

[objcl]
        ( [objcl_bakery] )
        ( [objcl_bank] )

[objcl_bakery]
        ( bakery )

CAN
        ( can )
        ( could )

CAN_YOU
        ( CAN you )

SHOW
        ( show )
        ( display )

TELL
        ( tell )
        ( explain *to )

GO
        ( get )
        ( go )

# ----------------------------------------
# greeting / farewell
#       hello
#       good bye
#       bye bye
# ----------------------------------------
s[greeting]
        ( [_NT_greeting] )

[_NT_greeting]
        ( hello )
        ( hi )

s[farewell]
        ( [_NT_farewell] )

[_NT_farewell]
        ( *good +bye )
\end{verbatim}

Non terminal symbols could either be surrounded by \texttt{[]} or could be
started with a capital letter. Terminal symbols have to be started
with a lower case letter. If you start a non terminal with a capital
our with the modifier \texttt{\_NT\_}, it is classified as an auxilliary non
terminal and will per default not occur in the parse tree. To express
optionality of a terminal or non terminal you have to use \texttt{*} and to
express repeatability you have to use \texttt{+} in front of a symbol. It is
also posible to combine optionality and repeatability by using \texttt{*+}.

Rules consist of a left hand side (LHS, the head of the rule) and a
right hand side (RHS, the body of the rule). If you want to use a rule
also as top level rule, i.e. a rule where you can start to parse from,
you have to put the modifier \texttt{s} in front of the rule. As you
can see above there are three top level rules:
\texttt{[request-path-description]}, \texttt{[greeting]} and
\texttt{[farewell]}. The lines in a RHS of a rule are interpreted as a
disjunction, the terminals and non terminals in one line as a
conjunction. It is not neccessary to define the rules in a special
order.

%%% Local Variables: 
%%% mode: latex
%%% TeX-master: t
%%% End: 

%% ========================================================================
%%  JANUS-SR   Janus Speech Recognition Toolkit
%%             ------------------------------------------------------------
%%             Object: Description of codebookSet
%%             ------------------------------------------------------------
%%
%%  Author  :  Florian Metze & many others
%%  Module  :  codebookSet.tex
%%  Date    :  $Id: codebookSet.tex 2390 2003-08-14 11:20:32Z fuegen $
%%
%%  Remarks :  
%%
%% ========================================================================
%%
%%   $Log$
%%   Revision 1.2  2003/08/14 11:18:58  fuegen
%%   Merged changes on branch jtk-01-01-15-fms (jaguar -> ibis-013)
%%
%%   Revision 1.1.2.2  2002/07/31 13:10:12  metze
%%   *** empty log message ***
%%
%%   Revision 1.1.2.1  2002/07/30 13:57:39  metze
%%   *** empty log message ***
%%
%% ========================================================================

\section{\Jlabel{file}{codebookSet}}

The description file read by \Jref{module}{CodebookSet}. An example
looks like:

\begin{verbatim}
; -------------------------------------------------------
;  Name            : codebookSetISLci
;  Type            : CodebookSet
;  Number of Items : 199
;  Date            : Thu Jul 11 20:21:13 2002
; -------------------------------------------------------
+QK-b           LDA                 48     32 DIAGONAL
+QK-m           LDA                 48     32 DIAGONAL
+QK-e           LDA                 48     32 DIAGONAL
SCH-b           LDA                 48     32 DIAGONAL
SCH-m           LDA                 48     32 DIAGONAL
SCH-e           LDA                 48     32 DIAGONAL
SIL-m           LDA                 48     32 DIAGONAL
T-b             LDA                 48     32 DIAGONAL
T-m             LDA                 48     32 DIAGONAL
T-e             LDA                 48     32 DIAGONAL
\end{verbatim}

The columns mean the codebook, the feature, the number of gaussians,
the number of dimensions and the covariance type.

%%% Local Variables: 
%%% mode: latex
%%% TeX-master: t
%%% End: 


\section{\Jlabel{file}{desc.tcl}}

A description file for a system. A typical file looks like:

{\small
\begin{verbatim}
# ========================================================================
#  JanusRTK   Janus Speech Recognition Toolkit
#             ------------------------------------------------------------
#             Object: System description
#             ------------------------------------------------------------
#
#  Author  :  Florian Metze
#  Module  :  desc.tcl
#  Date    :  $Id: desc.tex 2857 2008-12-09 10:02:33Z wolfel $
#
#  Remarks :  This is the description file for the ISLci system
#
# ========================================================================
# 
#  $Log$
#  Revision 1.2  2003/08/14 11:19:43  fuegen
#  Merged changes on branch jtk-01-01-15-fms (jaguar -> ibis-013)
#
#  Revision 1.1.2.7  2003/08/13 14:27:19  fuegen
#  formattings
#
#  Revision 1.1.2.6  2003/08/13 14:13:46  fuegen
#  readded definitions for CFGs
#
#  Revision 1.1.2.5  2003/08/11 12:41:08  soltau
#  windows support
#
# ========================================================================

# to make some scripts happy
set host [info hostname]
set pid  [pid]

# ------------------------------------------------------------------------
#  System and Path Definitions
# ------------------------------------------------------------------------

set SID                     ISLci

set projectHome             /home/njd/IslData
set ${SID}(path)            /home/njd/IslSystem/${SID}
set ${SID}(descPath)        [file join [set ${SID}(path)] desc]
set ${SID}(dictPath)        $projectHome
set ${SID}(lmPath)          $projectHome
set ${SID}(cfgPath)         $projectHome


# ------------------------------------------------------------------------
#  Welcome
# ------------------------------------------------------------------------

writeLog stderr "      ------ System $SID -----"
writeLog stderr "${argv0} reads desc.tcl: on $env(HOST).[pid], [exec date]"
writeLog stderr "using lib: $auto_path"


# ------------------------------------------------------------------------
#  Database
# ------------------------------------------------------------------------

set ${SID}(dbaseName)       db
set ${SID}(dbasePath)       $projectHome


# ------------------------------------------------------------------------
#  Phones & Tags
# ------------------------------------------------------------------------

set ${SID}(phonesSetDesc)    [set ${SID}(descPath)]/phonesSet
set ${SID}(tagsDesc)         [set ${SID}(descPath)]/tags


# ------------------------------------------------------------------------
#  Feature Set
# ------------------------------------------------------------------------

set ${SID}(testFeatureSetDesc)   @[file join [set ${SID}(descPath)] featDesc.test]
set ${SID}(meanFeatureSetDesc)   @[file join [set ${SID}(descPath)] featDesc.mean]
set ${SID}(featureSetDesc)       @[file join [set ${SID}(descPath)] featDesc]
set ${SID}(featureSetAccess)     @[file join [set ${SID}(descPath)] featAccess]
set ${SID}(featureSetLDAMatrix)   [file join [set ${SID}(path)] train lda${SID}.bmat]
set ${SID}(warpFile)             ""
set ${SID}(warpPhones)           "STIMMHAFT"
set ${SID}(meanPath)              [file join [set ${SID}(path)] train means]


# ------------------------------------------------------------------------
#  Stream: Codebook, Distribution, Tree
# ------------------------------------------------------------------------

set ${SID}(codebookSetDesc)  [file join [set ${SID}(descPath)] codebookSet]
set ${SID}(codebookSetParam) [set ${SID}(path)]/train/Weights/4.cbs.gz
set ${SID}(distribSetDesc)   [file join [set ${SID}(descPath)] distribSet]
set ${SID}(distribSetParam)  [set ${SID}(path)]/train/Weights/4.dss.gz
set ${SID}(padPhone)         @
set ${SID}(ptreeSetDesc)     ""
set ${SID}(distribTreeDesc)  [file join [set ${SID}(descPath)] distribTree]


# ------------------------------------------------------------------------
#  Transition models, topology and duration modelling
# ------------------------------------------------------------------------

set ${SID}(durSetDesc)       ""
set ${SID}(durPTreeDesc)     ""
set ${SID}(durTreeDesc)      ""

set ${SID}(tmDesc)           [set ${SID}(descPath)]/tmSet
set ${SID}(topoSetDesc)      [set ${SID}(descPath)]/topoSet
set ${SID}(ttreeDesc)        [set ${SID}(descPath)]/topoTree


# ------------------------------------------------------------------------
#  LM, Dictionary and Vocabulary
# ------------------------------------------------------------------------

set ${SID}(dictDesc)         [set ${SID}(dictPath)]/dict.50phones
set ${SID}(useXwt)           1
set ${SID}(optWord)          \$
set ${SID}(variants)         1

# -------------------------------------------------------
# Context Free Grammars
# -------------------------------------------------------

set cfgPath                  [set ${SID}(cfgPath)]
set ${SID}(cfg,grammars)     [list [list NAV \
                                        $cfgPath/cfg.ka.nav \
                                        $cfgPath/cfg.base.nav] \
                                   [list SHARED \
                                        $cfgPath/cfg.shared]]

# ------------------------------------------------------------------------
#  Testing 
# ------------------------------------------------------------------------

set ${SID}(testDictDesc)     [set ${SID}(dictDesc)]
set ${SID}(vocabDesc)        [set ${SID}(lmPath)]/vocab.germNews
set ${SID}(lmDesc)           [set ${SID}(lmPath)]/sz.ibis.gz
set ${SID}(ngramLMsegSize)   6
set ${SID}(lmWeight)         32
set ${SID}(lmPenalty)        3
set ${SID}(bbiSetDesc)       ""
set ${SID}(bbiSetParam)      ""


# ------------------------------------------------------------------------
#  Label Path
# ------------------------------------------------------------------------
 
set ${SID}(labelPath)     {/home/njd/IslSystem/ISLinit/labels/$spk/$utt.lbl}


set ${SID}(SPK)                  SPK     ; # speaker key
set ${SID}(UTT)                  UTTS    ; # utt     key
set ${SID}(TRL)                  TEXT    ; # trl     key
\end{verbatim}
}


{\tt  desc.tcl} also is   a good  place    to re-define other   common
functions such as {\tt  dbaseUttFilter} or {\tt hmmMakeUtterance}.  In
principle, you are free to re-configure everything  in this script, it
is however common  practice, to set  the Tcl-variable {\tt SID} to the
name  of  the directory, in which   this incarnation of {\tt desc.tcl}
resides.

%% ========================================================================
%%  JANUS-SR   Janus Speech Recognition Toolkit
%%             ------------------------------------------------------------
%%             Object: Description of dictionary
%%             ------------------------------------------------------------
%%
%%  Author  :  Florian Metze & many others
%%  Module  :  dictionary.tex
%%  Date    :  $Id: dictionary.tex 2390 2003-08-14 11:20:32Z fuegen $
%%
%%  Remarks :  
%%
%% ========================================================================
%%
%%   $Log$
%%   Revision 1.2  2003/08/14 11:18:58  fuegen
%%   Merged changes on branch jtk-01-01-15-fms (jaguar -> ibis-013)
%%
%%   Revision 1.1.2.1  2002/10/04 13:30:57  metze
%%   More docu
%%
%%
%% ========================================================================

\section{\Jlabel{file}{dictionary}}

A \Jref{module}{Dictionary} description file. It contains phones and tags.

An examples looks like this:

\begin{verbatim}
; ----------------------------
;   Example Dictionary
; ----------------------------
{$I} {{AY WB}}
{$} {{SIL WB}}
{(} {{SIL WB}}
{)} {{SIL WB}}
{Anne} {{AE WB} N {EH WB}}
{Anne(2)} {{AA WB} {N WB}}
\end{verbatim}

%%% Local Variables: 
%%% mode: latex
%%% TeX-master: t
%%% End: 

%% ========================================================================
%%  JANUS-SR   Janus Speech Recognition Toolkit
%%             ------------------------------------------------------------
%%             Object: Description of distribSet
%%             ------------------------------------------------------------
%%
%%  Author  :  Florian Metze & many others
%%  Module  :  distribSet.tex
%%  Date    :  $Id: distribSet.tex 2390 2003-08-14 11:20:32Z fuegen $
%%
%%  Remarks :  
%%
%% ========================================================================
%%
%%   $Log$
%%   Revision 1.2  2003/08/14 11:18:59  fuegen
%%   Merged changes on branch jtk-01-01-15-fms (jaguar -> ibis-013)
%%
%%   Revision 1.1.2.2  2002/07/31 13:10:12  metze
%%   *** empty log message ***
%%
%%   Revision 1.1.2.1  2002/07/30 13:57:39  metze
%%   *** empty log message ***
%%
%% ========================================================================

\section{\Jlabel{file}{distribSet}}

The description file used in a \Jref{module}{DistribSet}. An example
looks like this:

\begin{verbatim}
; -------------------------------------------------------
;  Name            : distribSetISLci
;  Type            : DistribSet
;  Number of Items : 199
;  Date            : Thu Jul 11 20:21:13 2002
; -------------------------------------------------------
+QK-b            +QK-b
+QK-m            +QK-m
+QK-e            +QK-e
SCH-b            SCH-b
SCH-m            SCH-m
SCH-e            SCH-e
SIL-m            SIL-m
T-b              T-b
T-m              T-m
T-e              T-e
\end{verbatim}

The second column tells you which codebook to use.

%%% Local Variables: 
%%% mode: latex
%%% TeX-master: t
%%% End: 

%% ========================================================================
%%  JANUS-SR   Janus Speech Recognition Toolkit
%%             ------------------------------------------------------------
%%             Object: Description of distribTree
%%             ------------------------------------------------------------
%%
%%  Author  :  Florian Metze & many others
%%  Module  :  distribTree.tex
%%  Date    :  $Id: distribTree.tex 2390 2003-08-14 11:20:32Z fuegen $
%%
%%  Remarks :  
%%
%% ========================================================================
%%
%%   $Log$
%%   Revision 1.2  2003/08/14 11:18:59  fuegen
%%   Merged changes on branch jtk-01-01-15-fms (jaguar -> ibis-013)
%%
%%   Revision 1.1.2.2  2002/07/31 13:10:12  metze
%%   *** empty log message ***
%%
%%   Revision 1.1.2.1  2002/07/30 13:57:39  metze
%%   *** empty log message ***
%%
%% ========================================================================

\section{\Jlabel{file}{distribTree}}

A \Jref{module}{Tree} description file, used for the distribution
tree. An example looks like this:

\begin{verbatim}
; -------------------------------------------------------
;  Name            : distribTreeISLci
;  Type            : Tree
;  Number of Items : 401
;  Date            : Thu Jul 11 20:21:13 2002
; -------------------------------------------------------
ROOT-b          {} ROOT-+QK-b ROOT-+QK-b ROOT-+QK-b -
ROOT-+QK-b      {0=+QK} ROOT-+hBR-b +QK-b - -
+QK-b           {} - - - +QK-b
ROOT-m          {} ROOT-+QK-m ROOT-+QK-m ROOT-+QK-m -
ROOT-+QK-m      {0=+QK} ROOT-+hBR-m +QK-m - -
+QK-m           {} - - - +QK-m
ROOT-e          {} ROOT-+QK-e ROOT-+QK-e ROOT-+QK-e -
ROOT-+QK-e      {0=+QK} ROOT-+hBR-e +QK-e - -
+QK-e           {} - - - +QK-e
ROOT-+hBR-b     {0=+hBR} ROOT-+hEH-b +hBR-b - -
+hBR-b          {} - - - +hBR-b
ROOT-+hBR-m     {0=+hBR} ROOT-+hEH-m +hBR-m - -
+hBR-m          {} - - - +hBR-m
ROOT-+hBR-e     {0=+hBR} ROOT-+hEH-e +hBR-e - -
+hBR-e          {} - - - +hBR-e
...
\end{verbatim}

%%% Local Variables: 
%%% mode: latex
%%% TeX-master: t
%%% End: 

%% ========================================================================
%%  JANUS-SR   Janus Speech Recognition Toolkit
%%             ------------------------------------------------------------
%%             Object: Description of featAccess
%%             ------------------------------------------------------------
%%
%%  Author  :  Florian Metze & many others
%%  Module  :  featAccess.tex
%%  Date    :  $Id: featAccess.tex 2390 2003-08-14 11:20:32Z fuegen $
%%
%%  Remarks :  
%%
%% ========================================================================
%%
%%   $Log$
%%   Revision 1.2  2003/08/14 11:18:59  fuegen
%%   Merged changes on branch jtk-01-01-15-fms (jaguar -> ibis-013)
%%
%%   Revision 1.1.2.2  2002/07/31 13:10:12  metze
%%   *** empty log message ***
%%
%%   Revision 1.1.2.1  2002/07/30 13:57:39  metze
%%   *** empty log message ***
%%
%% ========================================================================

\section{\Jlabel{file}{featAccess}}

This tells the \Jref{file}{featDesc} where to find the data. An
example looks like this:

\begin{verbatim}
set      adcfile      [file join /project/florian/isldata/adcs $arg(ADC)]
set      accessList   $sampleList
lappend  accessList  "ADCFILE $adcfile"
\end{verbatim}

%%% Local Variables: 
%%% mode: latex
%%% TeX-master: t
%%% End: 

%% ========================================================================
%%  JANUS-SR   Janus Speech Recognition Toolkit
%%             ------------------------------------------------------------
%%             Object: Description of featDesc
%%             ------------------------------------------------------------
%%
%%  Author  :  Florian Metze & many others
%%  Module  :  featDesc.tex
%%  Date    :  $Id: featDesc.tex 2390 2003-08-14 11:20:32Z fuegen $
%%
%%  Remarks :  
%%
%% ========================================================================
%%
%%   $Log$
%%   Revision 1.2  2003/08/14 11:18:59  fuegen
%%   Merged changes on branch jtk-01-01-15-fms (jaguar -> ibis-013)
%%
%%   Revision 1.1.2.4  2002/11/19 13:23:30  metze
%%   Beautification
%%
%%   Revision 1.1.2.3  2002/11/19 09:17:44  fuegen
%%   minor changes for overfull hboxes
%%
%%   Revision 1.1.2.2  2002/07/31 13:10:12  metze
%%   *** empty log message ***
%%
%%   Revision 1.1.2.1  2002/07/30 13:57:39  metze
%%   *** empty log message ***
%%
%% ========================================================================

\section{\Jlabel{file}{featDesc}}

The feature description file, read by the
\Jref{module}{FeatureSet}. An example looks like this:

\begin{verbatim}
# =======================================================
#  JanusRTk     Janus Recognition Toolkit
#               -----------------------------------------
#               Object: Feature Description
#               -----------------------------------------
#
#  Author    :  Hagen Soltau
#  Module    :  featDesc
#  Remarks   :  based on Hua's new frontend, 40 dimensions
#
#  $Log$
#  Revision 1.2  2003/08/14 11:18:59  fuegen
#  Merged changes on branch jtk-01-01-15-fms (jaguar -> ibis-013)
#
#  Revision 1.1.2.4  2002/11/19 13:23:30  metze
#  Beautification
#
#  Revision 1.1.2.3  2002/11/19 09:17:44  fuegen
#  minor changes for overfull hboxes
#
#  Revision 1.1.2.2  2002/07/31 13:10:12  metze
#  *** empty log message ***
#
#  Revision 1.1.2.1  2002/07/30 13:57:39  metze
#  *** empty log message ***
#
#  Revision 1.1  2002/03/04 16:10:49  soltau
#  Initial revision
#
# =======================================================

global WARPSCALE warpScales meanPath
global WAVFILE OLDSPK sas pms


# -------------------------------------------------------
#  Load Mean Vectors
# -------------------------------------------------------

if {![info exist OLDSPK] || $OLDSPK != $arg(spk) } {

  if {[llength [info command ${fes}Mean]]} { 
    ${fes}Mean  destroy
    ${fes}SMean destroy
  }

  if {[file exist $meanPath/$arg(spk).mean]} {
    FVector ${fes}Mean  13
    FVector ${fes}SMean 13
    writeLog stderr "$fes Loading $meanPath/$arg(spk).mean"
    ${fes}Mean  bload $meanPath/$arg(spk).mean
    ${fes}SMean bload $meanPath/$arg(spk).smean
    set OLDSPK $arg(spk)
  } else {
    writeLog stderr "$fes Loading $meanPath/$arg(spk).mean FAILED"
  }
}


# -------------------------------------------------------
#  Load ADC segment...
# -------------------------------------------------------

if {![info exist WAVFILE] || $WAVFILE != $arg(ADCFILE)} {

  set WAVFILE $arg(ADCFILE)

  if {[file exist $arg(ADCFILE).shn]} {
    $fes    readADC         ADC             $arg(ADCFILE).shn \
      -h 0 -v 0 -offset mean -bm shorten
  } else {
    $fes    readADC         ADC             $arg(ADCFILE)     \
      -h 0 -v 0 -offset mean -bm auto
  }

  $fes spectrum FFT     ADC 20ms
}


# -------------------------------------------------------
#  Get warp
# -------------------------------------------------------

if {![info exist WARPSCALE]} { 
  if [info exist warpScales($arg(spk))] { 
     set WARP $warpScales($arg(spk))
  }  else {
     set WARP 1.00
  } 
} else { set WARP $WARPSCALE } 

writeLog stderr "$fes ADCfile $arg(utt) WARP $WARP"


# -------------------------------------------------------
#  Vocal Tract Length Normalization + MCEP
# -------------------------------------------------------

$fes VTLN     WFFT      FFT  $WARP -mod lin -edge 0.8

if { [llength [objects FBMatrix matrixMEL]] != 1} {
   set melN 30
   set points [$fes:FFT configure -coeffN]
   set rate   [expr 1000 * [$fes:FFT configure -samplingRate]]
   [FBMatrix matrixMEL] mel -N $melN -p $points -rate $rate
}

$fes   filterbank       MEL             WFFT           matrixMEL
$fes   log              lMEL            MEL            1.0 1.0

set cepN 13

if { [llength [objects FMatrix matrixCOS]] != 1} {
   set n [$fes:lMEL configure -coeffN]
   [FMatrix matrixCOS] cosine $cepN $n -type 1
}

$fes   matmul           MCEP            lMEL            matrixCOS


# -------------------------------------------------------
#  Mean Subtraction, Delta, Delta-Delta and LDA
# -------------------------------------------------------

$fes meansub  FEAT    MCEP  -a 2 -mean ${fes}Mean -smean ${fes}SMean
$fes adjacent FEAT+   FEAT  -delta 5

if { [$fes index LDAMatrix] > -1} {
    $fes matmul   LDA     FEAT+ $fes:LDAMatrix.data -cut 32
}

if [info exists pms] {
    foreach p [$pms] {
        $fes matmul OFS-$p LDA $pms:$p.item(0)
        if [info exists sas] {
            $sas adapt $fes:OFS-$p.data $fes:OFS-$p.data 0
        }
    }
} else {
    if [info exists sas] {
        $sas adapt $fes:LDA.data $fes:LDA.data 0
    }
}

\end{verbatim}

Errors in the featDesc are not always easy to track. A much-used
strategy to debug errors in the featDesc is to plaster it with
\texttt{puts ``I am here''} commands, to find out where exactly in the
code the offending operation occurs.


%%% Local Variables: 
%%% mode: latex
%%% TeX-master: t
%%% End: 


\section{\Jindex{.janusrc}} \label{file:.janusrc}

This  describes the    file \Jindex{.janusrc},   which  is the    main
configuration file  for Janus.  A copy of  this file  can be  found in
\texttt{\~{ }/janus/scripts/janusrc}. It is  usable for both OSs, Unix
and Windows.

{\small
\begin{verbatim}
# =======================================================
#  JanusRTK   Janus Speech Recognition Toolkit
#             -------------------------------------------
#             Object: .janusrc - Resources file
#             -------------------------------------------
#
#  Author  :  Florian Metze and Christian Fuegen
#  Module  :  ~/.janusrc
#  Date    :  2000-08-07
#
#  Remarks :  This file is read in by janus on startup
#
#             It contains a few settings and redefines some
#             functions for compatibility with Linux and Windows
#
#             Original by Martin Westphal,
#             Dec. 4, 1996 for Janus3.2
#
# =======================================================
#  RCS info: $Id: janusrc.tex 3283 2010-07-02 17:46:12Z metze $
# 
#  $Log$
#  Revision 1.5  2007/02/23 10:15:13  fuegen
#  JANUSHOME can now be defined externally
#
#  Revision 1.4  2003/08/25 16:09:55  soltau
#  Added library path to auto_path
#
#  Revision 1.3  2003/08/18 13:03:36  soltau
#  removed some windows specific proc's (supported now by cli)
#
#  Revision 1.2  2003/08/14 11:19:43  fuegen
#  Merged changes on branch jtk-01-01-15-fms (jaguar -> ibis-013)
#
#  Revision 1.1.2.12  2003/08/13 09:41:01  soltau
#  final fixes
#
#  Revision 1.1.2.11  2003/08/12 16:12:37  metze
#  Cleanup for P013
#
#  Revision 1.1.2.10  2003/08/11 15:09:26  soltau
#  made GLOBALFP global
#
#  Revision 1.1.2.9  2003/08/11 14:29:32  soltau
#  exec windows support
#
#  Revision 1.1.2.8  2003/08/11 12:24:08  soltau
#  Windows fix for writing log-files:
#    set LOGFILE "janus.log" to pipe stdout from 'puts' to file
#
#  Revision 1.1.2.6  2003/06/26 15:09:20  metze
#  Changes for V5.0 P013
#
#  Revision 1.1.2.5  2003/04/30 15:42:00  metze
#  Final team
#
#  Revision 1.1.2.4  2003/04/09 14:42:05  metze
#  Typo fixed
#
#  Revision 1.1.2.3  2003/04/09 14:41:51  metze
#  Switched ngets off by default
#
#  Revision 1.1.2.2  2003/04/09 13:22:45  metze
#  Cleaned up ngets stuff
#
#  Revision 1.2  2003/01/17 15:42:24  fuegen
#  Merged changes on branch jtk-01-01-15-fms
#
#  Revision 1.1.2.1  2002/11/15 14:33:13  fuegen
#  Initial version
#
#
# =======================================================

# -------------------------------------------------------
#  check host and home
# -------------------------------------------------------

# for Condor & SLURM this is unreliable, therefore always set env(HOST)
set env(HOST) [lindex [split [info hostname] .] 0]

if {![info exists env(HOME)]} {
    set env(HOME) "HOME"
    #puts "set home directory : $env(HOME)"
} 

# -------------------------------------------------------
#  Set the auto path so that tcl libraries can be found.
# -------------------------------------------------------

if {[info exists env(JANUSHOME)]} { 
    set JANUSHOME $env(JANUSHOME)
} else {
    # E.g. for Windows:
    # set JANUSHOME "e:/ISL/hagen"

    # For Unix:
    set JANUSHOME  [file join $env(HOME) janus]
}

set JANUSLIB                        [file join $JANUSHOME gui-tcl]
set auto_path [linsert $auto_path 0 [file join $JANUSHOME tcl-lib]]
set auto_path [linsert $auto_path 0 [file join $JANUSHOME library]]
set auto_path [linsert $auto_path 0 $JANUSLIB]
regsub -all {\\} $auto_path {/} auto_path

# ----------------------------------------------------------------
#  WINDOWS dependent settings
#  1. define global variable LOGFILE to pipe stdout/stderr to file
#  2. manual sourcing of tcl-lib and gui-tcl
#  3. function redefinitions
#     exit  - for logging
#     puts  - output teeing into logfiles
#     exec  - to support some unix commands also under windows
# ----------------------------------------------------------------

if {[regexp {indows} $tcl_platform(os)]} {
    # uncomment this to pipe stdout/stderr to file
    # set LOGFILE "janus.log"

    # auto-sourcing
    set flist [concat [glob $JANUSHOME/gui-tcl/*.tcl] [glob $JANUSHOME/tcl-lib/*.tcl]]
    foreach f $flist { 
	if [string match "*makeIndex*" $f] continue
	if [string match "*JRTk*" $f]      continue
	if [string match "*test*" $f]      continue
	catch {source $f}
    }
    catch { rename exit exit-org }
    proc exit { args } {
	global GLOBALFP
	if [info exists GLOBALFP] { close $GLOBALFP }
	exit-org
    }

    catch { rename puts puts-org }
    proc puts { args } {
	global LOGFILE GLOBALFP
	set argc [llength $args]
	if {! [info exists LOGFILE] } { 
	    return [eval "puts-org $args"]
	}
	if {! [info exists GLOBALFP]} { set GLOBALFP [open $LOGFILE w] }
	set fp $GLOBALFP
	if {"-nonewline" == [lindex $args 0]} {
	    if {$argc == 3 } { set fp [lindex $args 1] }
	    if {$fp == "stdout" || $fp == "stderr"} { set fp $GLOBALFP}
	    puts-org -nonewline $fp [lindex $args end]
	} else {
	    if {$argc == 2} { set fp [lindex $args 0] }
	    if {$fp == "stdout" || $fp == "stderr"} { set fp $GLOBALFP}
	    puts-org $fp [lindex $args end]
	}
	return
    }

    catch { rename exec exec-org }
    proc exec { args } {
	global LOGFILE
        set cmd  [lindex $args 0]
        set opts [lrange $args 1 end]
	set cmdX [lsearch [list touch rm mkdir touch date] $cmd]

	if { $cmdX >= 0} { return [eval "$args"] }

	if { [catch {set res [eval exec-org $args]} msg] } {
	    # write error message to log file
	    if [info exists LOGFILE] {
		puts "ERROR pseudo-exec: \n called '$args' \n and got \n '$msg'\n"
	    }
	    error "ERROR pseudo-exec: \n called '$args' \n and got \n '$msg'\n"
	} else {
	    return $res
	}
    }
}

# ----------------------------------------------------------------
# Unix dependent settings
#  - socket based redefinitions of fgets and ngets 
#  - define socket host and port number
#  - start NGETS server via tcl-lib/ngetGUI.tcl 
#
# ----------------------------------------------------------------

if {0 && ![regexp {indows} $tcl_platform(os)]} {
    if {![info exists NGETS(HOST)]} {
	set NGETS(HOST)         islpc13
	set NGETS(PORT)         63060	
	set NGETS(VERBOSE)      1
	set NGETS(MGETS)        0
	
	catch {
	    regexp {uid=(\d+)} [exec id] dummy NGETS(PORT)
	    set NGETS(PORT) [expr $NGETS(PORT) + 52000]
	    unset dummy
	}
    }

    if {[regexp "^isl" $env(HOST)] && [string length $NGETS(HOST)] &&
	[string compare $env(HOST) $NGETS(HOST)]} {

	set NGETS(STARTUP) "using ngets: $NGETS(HOST):$NGETS(PORT)"

	# --------------------------------------
	#   FGETS from server
	# --------------------------------------
	catch {rename fgets fgets-org}
	proc fgets {file line_ } {
	    upvar  $line_ line
	    global NGETS
	    
	    if {[file pathtype $file] == "relative"} {
		set file "[pwd]/$file"
	    }
	    regsub -all "^/net"    $file ""         file
	    regsub -all "^/export" $file "/project" file
	    
	    return [ngets $file line]
	}
	
	# --------------------------------------
	#   GLOB from server, too
	# --------------------------------------
	catch {rename glob glob-org}
	proc glob { args } {
	    global NGETS
	    
	    set line ""
	    set nc   [regsub -- "-nocomplain " $args "" args]
	    regsub -- "--" $args "" args
	    
	    foreach f $args {
		set rel 0
		if {[file pathtype $f] == "relative"} {
		    set f [file join [pwd] $f]
		    set rel 1
		}
		
		# Strip '/net' from filenames
		regsub -all "^/net" $f "" f
		
		# Local filesystems don't need nglob
		if {[regexp "^/export" $f] || [regexp "^/tmp" $f]} {
		    set tmp [glob-org -nocomplain -- $f]
		} else {
		    set tmp [nglob $f]
		}
		if {$rel} {regsub -all " [pwd]/" " $tmp" " " tmp}
		append line [string trim $tmp]
	    }
	    
	    return $line
	}
    }
}

# ------------------------------------------------------------------------
#  Set the audio device for featshow.
# ------------------------------------------------------------------------

switch {tcl_platform(os)} {
    SunOS {
	set DEVICE SUN
	#set USERDEVICE {exec aplay -file $filename -g $gaindb -e int}
    }
    Linux {
	set DEVICEPLAY(User) {exec sox -q -t raw -r $rate -s -w $filename -t ossdsp -s -w /dev/dsp}
    }
}

# -------------------------------------------------------
#  General stuff
# -------------------------------------------------------

proc general_info {} {
   global tcl_platform tcl_version tk_version tcl_precision
   catch {puts "machine: $tcl_platform(machine) \
                         $tcl_platform(os) \
                         $tcl_platform(osVersion)"}
   catch {puts "tcl $tcl_version"}
   catch {puts "tk $tk_version"}
   catch {puts "tcl_precision: $tcl_precision"}   
}

proc writeJanusLog msg {
    global env
    puts stdout $msg
    flush stdout
}

catch { randomInit [pid] }

if {!$tcl_interactive} {
    set clicksatstart [clock clicks -milliseconds]
    catch { rename exit exit-org }
    proc exit { args } {
	global clicksatstart env
	set wt [expr .001*([clock clicks -milliseconds]-$clicksatstart)]
	puts stderr "ended [info nameofexecutable]: $env(HOST).[pid], [clock format [clock seconds]], system= [systemtime -total 1] user= [usertime -total 1] wall= $wt ([format %3.1f [expr 100.0*[usertime -total 1]/$wt]]%) "
	exit-org
    }
}

# -------------------------------------------------------
#  print start-up message
# -------------------------------------------------------

if {!$tcl_interactive} {
    puts stderr "started [info nameofexecutable]: $env(HOST).[pid], [clock format [clock seconds]]"
    #puts stderr "script: [info script] directory [pwd]"
    puts stderr "library: $auto_path"
}

if {[info exists NGETS(STARTUP)]} {
    puts $NGETS(STARTUP)
}
\end{verbatim}
}


It is read by JANUS at start-up. You'll then have to set your environment
variables correctly. Just for reference, my \texttt{.tcshrc} contains
the following Janus-related entries:

{\small
\begin{verbatim}
# For Janus:
setenv JANUS_LIBRARY $HOME/janus/library
setenv TCL_LIBRARY   /usr/lib/tcl8.3
setenv TK_LIBRARY    /usr/lib/tk8.3

# Compiling:
setenv IA32ROOT             /home/njd/intel/compiler60/ia32
setenv LD_LIBRARY_PATH      ${IA32ROOT}/lib

\end{verbatim}
}

For Windows, you should set the following environment variables, if
not already specified:

{\small
\begin{verbatim}
# take care of the '/' and '\'
HOME          C:\user\fuegen
JANUS_LIBRARY C:/user/fuegen/janus/library
\end{verbatim}
}

%% ========================================================================
%%  JANUS-SR   Janus Speech Recognition Toolkit
%%             ------------------------------------------------------------
%%             Object: Description of phonesSet
%%             ------------------------------------------------------------
%%
%%  Author  :  Florian Metze & many others
%%  Module  :  phonesSet.tex
%%  Date    :  $Id: phonesSet.tex 2390 2003-08-14 11:20:32Z fuegen $
%%
%%  Remarks :  
%%
%% ========================================================================
%%
%%   $Log$
%%   Revision 1.2  2003/08/14 11:19:00  fuegen
%%   Merged changes on branch jtk-01-01-15-fms (jaguar -> ibis-013)
%%
%%   Revision 1.1.2.4  2002/11/19 13:23:17  metze
%%   Beautification
%%
%%   Revision 1.1.2.3  2002/11/19 09:17:44  fuegen
%%   minor changes for overfull hboxes
%%
%%   Revision 1.1.2.2  2002/07/31 13:10:12  metze
%%   *** empty log message ***
%%
%%   Revision 1.1.2.1  2002/07/30 13:57:39  metze
%%   *** empty log message ***
%%
%% ========================================================================

\section{\Jlabel{file}{phonesSet}}

The phones that can be used. An example looks like:
% I IR IE IHR J K L M N NG O OR OE OEH ANG OH OHR P R S SCH T TS TSCH U UR UE UEH UEHR UH UHR V Z SIL +QK +hBR +hEH +hEM +hGH +hHM +hLG +hSM +nGN +nKL +nMK

\begin{verbatim}
PHONES    @ A AR AEH AEHR AH AHR AI AU B CH X D E E2 EH EHR ER ER2 EU F G
          I IR IE IHR J K L M N NG O OR OE OEH ANG OH OHR P R S SCH T TS 
          TSCH U UR UE UEH UEHR UH UHR V Z SIL +QK +hBR +hEH +hEM +hGH 
          +hHM +hLG +hSM +nGN +nKL +nMK
SILENCES  SIL
NOISES    +QK +hBR +hEH +hEM +hGH +hHM +hLG +hSM +nGN +nKL +nMK
AFFRIKATE TS TSCH
VOICED    M N NG L R A AEH AH E E2 EH ER2 I IE O OE OEH ANG OH U UE UEH UH
\end{verbatim}

The first item in  each line is the name  of a ``group'' of phones  in
the  set, while the remaining    items are phones. ``PHONES''   should
contain all phones.  Here, ``VOICED'' is  used for VTLN. ``AFFRIKATE''
and  ``VOICED'', ``NOISES'' and  ``SILENCES'' can be used as questions
during  context  clustering. ``@''  is  the  pad-phone, which is  used
whenever there is no context available.

%%% Local Variables: 
%%% mode: latex
%%% TeX-master: t
%%% End: 

%% ========================================================================
%%  JANUS-SR   Janus Speech Recognition Toolkit
%%             ------------------------------------------------------------
%%             Object: Description of ptree
%%             ------------------------------------------------------------
%%
%%  Author  :  Florian Metze & many others
%%  Module  :  ptreeSet.tex
%%  Date    :  $Id: ptreeSet.tex 2390 2003-08-14 11:20:32Z fuegen $
%%
%%  Remarks :  
%%
%% ========================================================================
%%
%%   $Log$
%%   Revision 1.2  2003/08/14 11:19:00  fuegen
%%   Merged changes on branch jtk-01-01-15-fms (jaguar -> ibis-013)
%%
%%   Revision 1.1.2.2  2002/07/31 13:10:12  metze
%%   *** empty log message ***
%%
%%   Revision 1.1.2.1  2002/07/30 13:57:39  metze
%%   *** empty log message ***
%%
%% ========================================================================

\section{\Jlabel{file}{ptreeSet}}

Used to define polyphone trees. An example looks like this:

\begin{verbatim}
; -------------------------------------------------------
;  Name            : distribTreeISLci
;  Type            : PTreeSet
;  Date            : Thu Jul 11 23:18:06 CEST 2002
; -------------------------------------------------------
+QK-b {+QK} 0 0 -count 1.000000 -model +QK-b
+QK-m {+QK} 0 0 -count 1.000000 -model +QK-m
+QK-e {+QK} 0 0 -count 1.000000 -model +QK-e
...
\end{verbatim}

%%% Local Variables: 
%%% mode: latex
%%% TeX-master: t
%%% End: 

%% ========================================================================
%%  JANUS-SR   Janus Speech Recognition Toolkit
%%             ------------------------------------------------------------
%%             Object: Description of search vocabulary
%%             ------------------------------------------------------------
%%
%%  Author  :  Florian Metze & many others
%%  Module  :  svocab.tex
%%  Date    :  $Id: svocab.tex 2390 2003-08-14 11:20:32Z fuegen $
%%
%%  Remarks :  
%%
%% ========================================================================
%%
%%   $Log$
%%   Revision 1.2  2003/08/14 11:19:01  fuegen
%%   Merged changes on branch jtk-01-01-15-fms (jaguar -> ibis-013)
%%
%%   Revision 1.1.2.2  2002/11/21 16:16:09  metze
%%   Pre-P012
%%
%%   Revision 1.1.2.1  2002/10/04 13:30:57  metze
%%   More docu
%%
%%
%% ========================================================================

\section{\Jlabel{file}{svocab}}

A \Jref{module}{SVocab} description file. It contains a list of words which
should also be contained in the dictionary.

An example looks like this:

\begin{verbatim}
$ 1
(
)
Anne
Anne(2)
\end{verbatim}

The ``1'' in the first line declares ``\$'' to be a filler-word, i.e. 
a word which is not handled by the language model. Instead, the \Jref{SVMap}{-filPen}
is added for every transition into this word. ``(`` and ``)'' are the begin-of-sentence
and end-of-sentence words.

%%% Local Variables: 
%%% mode: latex
%%% TeX-master: t
%%% TeX-master: t
%%% End: 

%% ========================================================================
%%  JANUS-SR   Janus Speech Recognition Toolkit
%%             ------------------------------------------------------------
%%             Object: Description of tags
%%             ------------------------------------------------------------
%%
%%  Author  :  Florian Metze & many others
%%  Module  :  tags.tex
%%  Date    :  $Id: tags.tex 2390 2003-08-14 11:20:32Z fuegen $
%%
%%  Remarks :  
%%
%% ========================================================================
%%
%%   $Log$
%%   Revision 1.2  2003/08/14 11:19:01  fuegen
%%   Merged changes on branch jtk-01-01-15-fms (jaguar -> ibis-013)
%%
%%   Revision 1.1.2.2  2002/07/31 13:10:12  metze
%%   *** empty log message ***
%%
%%   Revision 1.1.2.1  2002/07/30 13:57:39  metze
%%   *** empty log message ***
%%
%% ========================================================================

\section{\Jlabel{file}{tags}}

A \Jref{module}{Tags} description file. It contains the modifiers for
phones that can be used in the \Jref{module}{Dictionary}.

An examples looks like:

\begin{verbatim}
WB
\end{verbatim}


%%% Local Variables: 
%%% mode: latex
%%% TeX-master: t
%%% End: 

%% ========================================================================
%%  JANUS-SR   Janus Speech Recognition Toolkit
%%             ------------------------------------------------------------
%%             Object: Description of tmSet
%%             ------------------------------------------------------------
%%
%%  Author  :  Florian Metze & many others
%%  Module  :  tmSet.tex
%%  Date    :  $Id: tmSet.tex 2390 2003-08-14 11:20:32Z fuegen $
%%
%%  Remarks :  
%%
%% ========================================================================
%%
%%   $Log$
%%   Revision 1.2  2003/08/14 11:19:01  fuegen
%%   Merged changes on branch jtk-01-01-15-fms (jaguar -> ibis-013)
%%
%%   Revision 1.1.2.2  2002/07/31 13:10:13  metze
%%   *** empty log message ***
%%
%%   Revision 1.1.2.1  2002/07/30 13:57:39  metze
%%   *** empty log message ***
%%
%% ========================================================================

\section{\Jlabel{file}{tmSet}}

The transition set description file. An example looks like:

\begin{verbatim}
SIL  { { 0  0.01 } { 1  0.0 } }
1    { { 0  0.01 } { 1  0.0 } }
3    { { 0  0.01 } { 1  0.0 } { 2  0.015 } }
\end{verbatim}

The Tcl-list contains the distance to transition (so ``0'' is a
self-loop) and the score for this transition.

%%% Local Variables: 
%%% mode: latex
%%% TeX-master: t
%%% End: 

%% ========================================================================
%%  JANUS-SR   Janus Speech Recognition Toolkit
%%             ------------------------------------------------------------
%%             Object: Description of topoSet
%%             ------------------------------------------------------------
%%
%%  Author  :  Florian Metze & many others
%%  Module  :  topoSet.tex
%%  Date    :  $Id: topoSet.tex 2390 2003-08-14 11:20:32Z fuegen $
%%
%%  Remarks :  
%%
%% ========================================================================
%%
%%   $Log$
%%   Revision 1.2  2003/08/14 11:19:02  fuegen
%%   Merged changes on branch jtk-01-01-15-fms (jaguar -> ibis-013)
%%
%%   Revision 1.1.2.2  2002/07/31 13:10:13  metze
%%   *** empty log message ***
%%
%%   Revision 1.1.2.1  2002/07/30 13:57:40  metze
%%   *** empty log message ***
%%
%% ========================================================================

\section{\Jlabel{file}{topoSet}}

The description file for a \Jref{module}{TopoSet}:

An example looks like this:

\begin{verbatim}
6state  { ROOT-b ROOT-b ROOT-m ROOT-m ROOT-e ROOT-e } { 3 3 3 3 3 3 }
3state  { ROOT-b ROOT-m ROOT-e } { 1 1 1 }
SIL     { ROOT-m ROOT-m ROOT-m ROOT-m } { 1 1 1 1 }
\end{verbatim}

The second colum defines the root-node for the model tree, while the
second column defines the transition to use from the \Jref{module}{TmSet}.

%%% Local Variables: 
%%% mode: latex
%%% TeX-master: t
%%% End: 

%% ========================================================================
%%  JANUS-SR   Janus Speech Recognition Toolkit
%%             ------------------------------------------------------------
%%             Object: Description of topoTree
%%             ------------------------------------------------------------
%%
%%  Author  :  Florian Metze & many others
%%  Module  :  topoTree.tex
%%  Date    :  $Id: topoTree.tex 2390 2003-08-14 11:20:32Z fuegen $
%%
%%  Remarks :  
%%
%% ========================================================================
%%
%%   $Log$
%%   Revision 1.2  2003/08/14 11:19:02  fuegen
%%   Merged changes on branch jtk-01-01-15-fms (jaguar -> ibis-013)
%%
%%   Revision 1.1.2.2  2002/07/31 13:10:13  metze
%%   *** empty log message ***
%%
%%   Revision 1.1.2.1  2002/07/30 13:57:40  metze
%%   *** empty log message ***
%%
%% ========================================================================

\section{\Jlabel{file}{topoTree}}

The description file for the topology tree, which can be read in a
in the \Jref{module}{Tree} object.

An example looks like this:

\begin{verbatim}
ROOT    { 0=SIL } 6state SIL    -       -
6state  {       }    -    -     -       3state
SIL     {       }    -    -     -       SIL
\end{verbatim}

It defines the topologies to use for different phones, defined by the
question in the second column (standard tree answer format: ``no, yes,
don't-know, leaf'' for colums 3-6).

%%% Local Variables: 
%%% mode: latex
%%% TeX-master: t
%%% TeX-master: t
%%% End: 


\section{\Jindex{db-spk}, \Jindex{db-utt}} \label{file:dbase}

Janus  contains a database   object which  stores all  the information
needed for  a particular system. An  example script to generate such a
dbase is available in \texttt{\~{ }/janus/scripts/genDBase.tcl}.

The database consists of two parts, each of which is store in a
data-file (*.dat) and an index file (*.idx):

\begin{description}
\item[db-spk] The ``speaker database''

  Every entry in  this database (corresponding to  a line in the file)
  contains information for one  ``speaker''. It should contain a field
  ``UTTS'', which lists all the  utterances (segments) which belong to
  this speaker. Also, paths to  ADC files, speaker information or warp
  factors can be stored here.

\item[db-utt] The ``utterance database''

  Every entry in this database  (corresponding to a  line in the file)
  contains  information for one   ``utterance''.  It should  contain a
  field ``SPK'', which links to the corresponding entry in the speaker
  database, a field ``UTT'', which  repeats the utterance id and further
  information (transliteration: ``TEXT'', ADC segment, ...)

\end{description}

Look at  \texttt{\~{ }/janus/scripts/genDBase.tcl} to see   how these
files can be generated from free-format data.

%%% Local Variables: 
%%% mode: latex
%%% TeX-master: t
%%% End: 
