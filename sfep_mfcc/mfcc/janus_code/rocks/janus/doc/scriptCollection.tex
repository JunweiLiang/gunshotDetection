%% ========================================================================
%%  JANUS-SR   Janus Speech Recognition Toolkit
%%             ------------------------------------------------------------
%%             Object: Documentation for Script Collection
%%             ------------------------------------------------------------
%%
%%  Author  :  Florian Metze & many others
%%  Module  :  scriptCollection.tex
%%  Date    :  $Id: scriptCollection.tex 2568 2004-11-09 19:54:57Z metze $
%%
%%  Remarks :
%%
%% ========================================================================
%%
%%   $Log$
%%   Revision 1.4  2004/11/09 19:53:48  metze
%%   *** empty log message ***
%%
%%   Revision 1.3  2004/08/16 16:10:13  metze
%%   Additions for P014 by Florian
%%
%%   Revision 1.2  2003/08/14 11:18:56  fuegen
%%   Merged changes on branch jtk-01-01-15-fms (jaguar -> ibis-013)
%%
%%   Revision 1.1.2.7  2003/08/07 13:01:24  metze
%%   *** empty log message ***
%%
%%   Revision 1.1.2.6  2003/04/30 15:43:59  metze
%%   Changes before the new repository
%%
%%   Revision 1.1.2.5  2002/11/21 16:15:46  metze
%%   Pre-P012
%%
%%   Revision 1.1.2.4  2002/11/05 13:43:49  soltau
%%   typo
%%
%%   Revision 1.1.2.3  2002/11/05 07:45:44  metze
%%   Removed reference to trainPT
%%
%%   Revision 1.1.2.2  2002/10/04 13:30:29  metze
%%   More docu
%%
%%   Revision 1.1.2.1  2002/08/26 16:40:37  metze
%%   Re-organization of modules
%%
%%
%% ========================================================================


\section{The Janus Library: ``tcl-lib'' and ``gui-tcl''}

The Tcl-library is  a set of procedures  the user can invoke and which
provide  a number of ``convenience''  functions.   The scripts in  the
Janus  Scripts Collection  (\texttt{\~{ }/janus/scripts},  see chapter
\Jref{janus}{scripts})  use      the    Tcl-library extensively.   The
Tcl-library   can  be   found   in   \texttt{\~{ }/janus/tcl-lib}  and
\texttt{\~{ }/janus/gui-tcl}.   To  auto-load   the  functions,  the
Tcl-variable \texttt{auto\_path} has to be set correctly, i.e.  to the
value of these two directories. Also, a  file \texttt{tclIndex} has to
exist   in these directories. You  should  not need to   worry, if you
follow  the standard install instructions,  otherwise refer to any Tcl
manual for a description of the auto-loading mechanism.  The functions
available in the tcl-lib are described in chapter \Jref{sec}{lib}.

\section{The Janus Scripts Collection} \label{janus:scripts}

The  directory  \texttt{\~{ }/janus/scripts}  contains  a  number  of
scripts, which   we  normally use  to train  and   test systems. These
scripts  are  often modified and    copied in a  system  directory for
documentation purposes.

If you have access to an example system  (i.e. IslData, IslSystem), we
suggest that  you have a  look at it  to see how  data and scripts are
typically  organized in a  JRTk project.  Usually,  the structure of a
project  looks as  follows (this  project  would be called the  ``M1''
system):

{\small
\begin{verbatim}
M1/
|
+--master.log
|
+--Log/
|  '--makeCI.log
|    
+--desc/
|  +--desc.tcl
|  +--codebookSet
|  +--distribSet
|  +--distribTree
|  +--featAccess
|  +--featDesc
|  +--phonesSet
|  +--tags
|  +--tmSet
|  +--topoSet
|  '--topoTree
|
+--train/
|  +--ldaM1.bmat
|  +--ldaM1.counts
|  +--convList
|  |
|  +--Log/
|  |  +--lda.log
|  |  +--samples.log
|  |  +--kmeans.log
|  |  '--train.log
|  |
|  +--Accus/
|  |  +- 1.cba.gz
|  |  '--1.dsa.gz
|  |
|  '--Weights
|     +--0.cbs.gz
|     '--0.dss.gz
|
+--test/
   +--convList
   |
   +--Log/
   |  '--test.log
   |
   '--hypos/
      '--H_kottan_z26_p0_LV.ctm
\end{verbatim}
}

Typically, a  system  directory (here:  ``M1'')  contains a number  of
sub-directories, each  for  different phases (label  writing, cepstral
mean computation,  model  training (``train''),   polyphone  training/
clustering, testing (``test''), ...). Each directory then contains the
data resulting from this step and log-files.

The   scripts   who   perform   the  operations   can   be left  under
\texttt{janus/scripts}, only \texttt{desc.tcl} is a configuration file
specific to  this project  and  is therefore copied into  the  project
directory along with the other description files.

\subsection{Available Scripts}

The following scripts are available in the Janus Scripts Collection
(in the order in which they are usually called):

\begin{description}

\Jitem{genDBase.tcl} \label{janus:genDBase.tcl}

  This script can be used to create a database, which is necessary for
  all further steps.  Look at the resulting  database files (they  are
  called \texttt{db-\{spk|utt\}.\{idx|dat\}} to see what   information
  can and needs to be defined in the database.

  Depending on your needs  and   the format, in   which you  have  the
  information  available, you will need  to modify this script to suit
  your needs.

\Jitem{makeCI.tcl}    \label{janus:makeCI.tcl}

  Creates the following description files for a context-independent
  (CI) system:

  \begin{itemize}
    \item codebookSet
    \item distribSet
    \item distribTree
  \end{itemize}
  
  You'll need to have all the other description files in place, namely
  the phonesSet. If you want to use a different architecture
  (i.e. semi-tied, or non-tri-state architectures), you can edit this
  file according to your needs.

\Jitem{means.tcl}     \label{janus:means.tcl}

  This script will create the cepstral means needed for the standard
  pre-processing of the Janus-based recognizers.

\Jitem{lda.tcl}       \label{janus:lda.tcl}

  This script computes an LDA matrix, used for the standard
  pre-processing.

\Jitem{samples.tcl}   \label{janus:samples.tcl}

  This script extracts samples for further clustering with \Jref{janus}{kmeans.tcl}.

\Jitem{kmeans.tcl}    \label{janus:kmeans.tcl}

  Performs KMeans clustering on data extracted with \Jref{janus}{samples.tcl}.

\Jitem{train.tcl}     \label{janus:train.tcl}

  Performs EM-training on initial codebooks from \Jref{janus}{kmeans.tcl}.
  Can be used for training of a context-independent (CI),
  a context-dependent (CD), or a polyphone (PT) system.
  Normally, we do label training although it is also possible
  to do viterbi- or forward-backward-training by replacing
  \Jref{proc}{viterbiUtterance} by for example \Jref{proc}{viterbiUtterance}.

\Jitem{makePT.tcl}    \label{janus:makePT.tcl}

  Creates a polyphone (PT) system from the CI description files.

\Jitem{cluster.tcl}   \label{janus:cluster.tcl}

  Clusters the contexts from PT training.

\Jitem{split.tcl}     \label{janus:split.tcl}

  Creates the context-dependent (CD) models after PT training, creates
  the following CD description files (with $N > 0$):

  \begin{itemize}
    \item codebookSet.N.gz
    \item distribSet.Np.gz
    \item distribTree.Np.gz
  \end{itemize}

\Jitem{createBBI.tcl} \label{janus:createBBI.tcl}

  Creates a BBI (bucket-box intersection) tree for a codebook.

\Jitem{test.tcl}      \label{janus:test.tcl}

  Tests a system.

\Jitem{score.tcl}     \label{janus:score.tcl}
  
  Scores a system, i.e. computes word error rates.

\Jitem{ana\_time.tcl}  \label{janus:anatime.tcl}

  Allows to  measure the CPU-time  spent in  pre-defined sections of a
  script. You can find more details about timing analysis and how to use
  this script in section \ref{ibis:timing}.

\Jitem{labels.tcl}    \label{janus:labels.tcl}

  Writes new labels with an existing system. Can also be used to
  bootstrap a new system using the acoustic models from another system.

\end{description}

An example  \texttt{desc.tcl}  file  is also  included in  the  script
collection.  All  ``working'' scripts source \texttt{../desc/desc.tcl}
and load the settings (paths, ...)  from there,  although these can be
overridden   at       the   command line        or   in   the   script
themself. \texttt{janusrc} is an example configuration file for janus,
which  is  best  adapted   and copied   into your home    directory as
\texttt{.janusrc}.

\subsection{Working with master.tcl} \label{janus:master.tcl}

We  assume  you have  a  system directory   setup correctly, including
pre-computed   time-alignments   (``labels''). When  working with  the
example system ``IslSystem'', you have a \texttt{desc} directory which
contains  an appropriate  \Jref{file}{desc.tcl}  file. In the ``system
home directory'' (``M1'' in the above example), you can now enter

\begin{verbatim}
janus <janus>/scripts/master.tcl -do init means lda samples kmeans train
\end{verbatim} 

and the master script will create a context-independent (CI) system in
the \texttt{M1/train}  directory.  \texttt{$<$janus$>$} refers to your
Janus  installation     directory.\footnote{Usually   this   will   be
\texttt{\~{ }/janus}.}   You'll find    logfiles  in   your system's
\texttt{Log}   subdirectory.    The following  steps  were  performed,
calling the following scripts:

\begin{enumerate}
\item[init] (\Jref{janus}{makeCI.tcl})
  to create the codebookSet, distribSet, distribTree definition files
  for the CI system

\item[means] (\Jref{janus}{means.tcl})
  to compute the cepstral means for this preprocessing. This can be
  re-used for a CD-system

\item[lda] (\Jref{janus}{lda.tcl})
  to compute the LDA (Linear Discriminant Analysis) matrix for this
  system.

\item[samples] (\Jref{janus}{samples.tcl})
  to extract samples for the CI models

\item[kmeans] (\Jref{janus}{kmeans.tcl})
  to create initial codebooks from the samples, written into {\tt Weights/0}.
  Once this step is completed, you can remove the {\tt data} subdirectory.

\item[train] (\Jref{janus}{train.tcl})
  to perform several iterations of EM-training on the initial codebooks.
  At the end of this step, you can remove the contents of the {\tt Accus}
  subdirectory as well as intermediate codebooks in {\tt Weights}, to save
  space.

\end{enumerate}

\Jref{janus}{master.tcl} will show you the  command lines it executes,
if you want   to parallelize your  training, you  can copy  the output
lines   {\tt exec  janus lda.tcl   ...} (omitting  the  {\tt exec} and
changing the   log  file name) and  run   the same  script  on several
machines.

To run the polyphone training, enter

\begin{verbatim}
janus <janus>/scripts/master.tcl -do makePT trainPT cluster split
\end{verbatim}

This will create the description files for a context-dependent system.

To run the training for the context-dependent system, enter

\begin{verbatim}
janus <janus>/scripts/master.tcl -do lda samples kmeans train test score
\end{verbatim}

assuming  that you have created  a new directory  and set up the paths
for codebookSetParam and distribSetParam accordingly. To create initial
time alignments for a new system, edit the description file (probably
you'll have to change most of the files usually in the ``desc'' directory
to match your old system and your desired new setup) and execute:

\begin{verbatim}
janus <janus>/scripts/master.tcl -do labels
\end{verbatim}

If you type

\begin{verbatim}
janus <janus>/scripts/master.tcl -h
\end{verbatim}

you'll get a list of all command line options for \texttt{master.tcl}.

\subsection{Extra scripts} \label{janus:extrascripts}

The   \texttt{scripts} directory  also  contains a  number of scripts,
which    can       not     (currently)    be       called      through
\Jref{janus}{master.tcl}. They  can however  serve as  example scripts,
which can be adapted to specific problems.

\begin{description}

\Jitem{map.tcl} \label{janus:map.tcl}

An example script to perform \Jgloss{MAP} adaptation.

\Jitem{mllr.tcl} \label{janus:mllr.tcl}

An example script to perform \Jgloss{MLLR} adaptation.

\end{description}

%%% Local Variables: 
%%% mode: latex
%%% TeX-master: t
%%% End: 
