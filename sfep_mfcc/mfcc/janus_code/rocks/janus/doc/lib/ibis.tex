%% ========================================================================
%%  JANUS-SR   Janus Speech Recognition Toolkit
%%             ------------------------------------------------------------
%%             Object: Documentation for ibis.tcl
%%             ------------------------------------------------------------
%%
%%  Author  :  Florian Metze & many others
%%  Module  :  ibis.tex
%%  Date    :  $Id: ibis.tex 2390 2003-08-14 11:20:32Z fuegen $
%%
%%  Remarks :  
%%
%% ========================================================================
%%
%%   $Log$
%%   Revision 1.2  2003/08/14 11:19:04  fuegen
%%   Merged changes on branch jtk-01-01-15-fms (jaguar -> ibis-013)
%%
%%   Revision 1.1.2.1  2002/08/27 16:19:27  metze
%%   New version of automatic generator.
%%
%%   Revision 1.1.2.3  2002/08/01 13:42:21  metze
%%   Fixes for clean documentation.
%%
%%   Revision 1.1.2.2  2002/07/31 13:10:12  metze
%%   *** empty log message ***
%%
%%   Revision 1.1.2.1  2002/05/28 14:12:31  metze
%%   Initial version of the Janus/ Ibis documentation
%%
%%
%% ========================================================================

\section{\Jlabel{lib}{ibis.tcl}}

This describes the tcl-lib  file \texttt{ibis.tcl}, which helps you to
use the \Jgloss{Ibis} decoder. It contains the following procedures:

\subsection{\Jlabel{proc}{printDo}}

Just a little helper routine, which will show a command on the screen and
then execute it.

\subsection{\Jlabel{proc}{ibisInit}}

Can be used to initialize a full decoder all in one command. It will do
the following steps:

\begin{enumerate}

\item Create \Jref{module}{PHMMSet}, \Jref{module}{LCMSet}, \Jref{module}{RCMSet}, and possibly \Jref{module}{XCMSet}

\item Create a \Jref{module}{SVocab}

\item Create a language model of  any type using \Jref{proc}{ibisCreateLM},
possibly also create a separate \Jref{module}{NGramLM} for the Look-Ahead
structure

\item Create a \Jref{module}{SVMap}

\item Read the vocabulary and initialize the mapper, unless we read a \Jref{module}{STree} dump file

\item  Create the \Jref{module}{STree},  maybe reading in a dump file,
create the \Jref{module}{LTree} and the \Jref{module}{SPass} object.

\end{enumerate}

As is good practice for xInit procedures, the parameters can be 
specified in one of three ways (in order of precedence):

\begin{enumerate}
\item On the command line.
\item In the \$SID array
\item By default values in the procedure
\end{enumerate}

If any of the needed  objects already exist,  they won't be recreated.
This  is very handy  because  you can build   for example your complex
language model yourself (possibly using  dumps where appropriate)  and
then call \Jref{proc}{ibisInit}, which will do all the remaining work.
Remember,  you   can  create    \Jref{module}{STree}  dumps  with  the
\texttt{-dumpLM 0} option to \Jref{STree}{dump}.

\begin{description}

\item[Parameters:] \texttt{ibisInit \$SID}

\begin{description}

\Jitem{\Jlabel{ibisInit}{-dict} \Jref{module}{Dictionary},  \Jlabel{ibisInit}{-ttree} \Jref{module}{Tree},
  \Jlabel{ibisInit}{-phmmSet} \Jref{module}{PHMMSet}, \Jlabel{ibisInit}{-lcmSet} \Jref{module}{LCMSet},}
  \texttt{\Jlabel{ibisInit}{-rcmSet} \Jref{module}{RCMSet},   \Jlabel{ibisInit}{-xcmSet} \Jref{module}{XCMSet},
  \Jlabel{ibisInit}{-vocab} \Jref{module}{SVocab},    \Jlabel{ibisInit}{-svmap} \Jref{module}{SVMap},
  \Jlabel{ibisInit}{-stree} \Jref{module}{STree},     \Jlabel{ibisInit}{-ltree} \Jref{module}{LTree},
  \Jlabel{ibisInit}{-lmla} \Jref{module}{NGramLM},  \Jlabel{ibisInit}{-lm}    lm}
  allow you to tell ibisInit which (existing) object to use. Standard values are e.g.
  \texttt{dict\$SID} and \texttt{ttree\$SID}. The language model used for the lookahead
  has to be of type \Jref{module}{NGramLM}.

\Jitem{\Jlabel{ibisInit}{-streeDump} dump,   \Jlabel{ibisInit}{-vocabDesc} vocab,
  \Jlabel{ibisInit}{-mapDesc} map,}          \texttt{\Jlabel{ibisInit}{-phraseDesc} phraseDesc,
  \Jlabel{ibisInit}{-baseLMDesc} baseLMDesc, \Jlabel{ibisInit}{-ipolLMDesc} ipolLMDesc,
  \Jlabel{ibisInit}{-lmDesc} lmDesc,         \Jlabel{ibisInit}{-lmlaDesc} lmlaDesc}
  allow you to specify the name of the description files for the objects. If \texttt{-streeDump}
  is specified, \texttt{vocab} will not be read!
  
\Jitem{\Jlabel{ibisInit}{-lalz} x, \Jlabel{ibisInit}{-lz} x, \Jlabel{ibisInit}{-lp} x,
  \Jlabel{ibisInit}{-masterBeam} x} will configure \Jref{module}{SVMap} and \Jref{module}{SPass}.

\Jitem{\Jlabel{ibisInit}{-cacheN} n, \Jlabel{ibisInit}{-depth} n} configure the \Jref{module}{LTree}.

\Jitem{\Jlabel{ibisInit}{-lmType} type} allows to set the type of the language model to be
  created.

\Jitem{\Jlabel{ibisInit}{-segSize} n} will be passed to \Jref{module}{NGramLM} when
  loading the language model.

\Jitem{\Jlabel{ibisInit}{-xcm} 0$|$1, \Jlabel{ibisInit}{-ignoreRCM} 0$|$1} configure the
  behavior of the \Jref{module}{XCMSet}, if one exists.

\Jitem{\Jlabel{ibisInit}{-fastMatch} LFID} configures the system ID for the fast match
  module, which you'll have to create for yourself.

\end{description}

\item[Prerequisites] \

It is best to call \Jref{proc}{ibisInit} in a sequence such as:

\begin{verbatim}

senoneSetInit   $SID  distribStream$SID
topoSetInit     $SID
ttreeInit       $SID 
dictInit        $SID
trainInit       $SID
dbaseInit       $SID  $dbase -path $dbasePath


# ------------------------------------------------
#  Init New Janus Decoder
#    The STree-dump contains the SenoneSet, too!!!
# ------------------------------------------------

ibisInit $SID -masterBeam $masterBeam -lz $lz -lp $lp
svmap$SID readLMMap
svmap$SID map class

\end{verbatim}

% $

\Jref{proc}{ibisInit} requires and will check for the following objects:

\begin{enumerate}

\item{\Jref{module}{Tree}} the topology tree.
\item{\Jref{module}{Dictionary}} the dictionary.

\end{enumerate}

\item[Description] \

In this example, ibisInit  loaded the language  model, too, but we are
free to reconfigure the \Jref{module}{SVMap} after the creation of the
search tree. The filenames  to use were  configured in the \$SID array
(please have a  look at the code or  existing systems to  see how that
works).

\end{description}

\subsection{\Jlabel{proc}{phraseLMReadMap}}

Can be used to read a standard map-file into a \Jref{module}{PhraseLM}.
The Ibis decoder no longer uses a true ``Map'' language model, but
splits its contents into two parts for greater efficiency:

\begin{enumerate}
\item The one-word parts usually can be treated very efficiently in the \Jref{module}{SVMap}
\item The multi-words will be handled by \Jref{module}{PhraseLM}
\end{enumerate}

However, there are situations when you want to handle word-to-word mappings
in a PhraseLM, too, for example when handling stacked language models that
involve pronunciation scores. This procedure deals with these issues and
works in the same way as \Jref{proc}{svmapReadMap}, also accessible as 
\Jref{PhraseLM}{readMapFile} and \Jref{SVMap}{readMapFile}.

A map-file contains lines of the form:

\begin{verbatim}
{AB(2)}      {AB}
{AND_HAD(1)} {AND HAD} -prob -0.943350
\end{verbatim}

The \texttt{-prob} part is optional.

\begin{description}

\item[Parameters:] \texttt{phraseLMReadMap file}

\begin{description}
\Jitem{\Jlabel{phraseLMReadMap}{-mode} base$|$multi$|$all:} do only
include the multi-words (``multi'') or only add baseforms (i.e. words that
don't contain a '(') to the language model (``base''); ``all'' includes all
words found in the map-file (useful only with cascaded language models)
\Jitem{\Jlabel{phraseLMReadMap}{-base} lmodel} language
model underlying the \Jref{module}{PhraseLM}, can be specified here, too
\Jitem{\Jlabel{phraseLMReadMap}{-verbose} 0$|$1} output
\end{description}

\item[Prerequisites] none really, apart from a \Jref{module}{PhraseLM}
  and a file ;-).

\end{description}

\subsection{\Jlabel{proc}{svmapReadMap}}

Pretty much the twin of \Jref{proc}{phraseLMReadMap}. Reads a map-file
and adds the mappings not covered by the underlying language model to
the \Jref{module}{SVMap}.

\begin{description}

\item[Parameters:] \texttt{svmapReadMap file}

\begin{description}
\Jitem{\Jlabel{svmapReadMap}{-lm} lmodel} language
model underlying the \Jref{module}{SVMap}, can be specified here, too
\Jitem{\Jlabel{svmapReadMap}{-verbose} 0$|$1} output
\end{description}

\item[Prerequisites] none really, apart from a \Jref{module}{SVMap}
  and a file ;-).

\item[Description] This procedure will add a mapping for a word, if
its pronunciation probability in an underyling language model
is different from the one stored in the file.

\end{description}

\subsection{\Jlabel{proc}{ibisCreateLM}}

Function mainly for internal use only, you can however use it to
construct the following language model types: \Jref{module}{PhraseLM},
\Jref{module}{MetaLM}, \Jref{module}{NGramLM} or
\Jref{module}{STree}, if the language model is in fact stored in the
STree's dump file (it will simply create an \Jref{module}{NGramLM}
but not load anything).

\subsection{\Jlabel{proc}{spassWriteHypo}}

Writes a hypo from a \Jref{module}{SPass} to file or stdout. The
format is CTM, hence this procedure is also 
available as \Jref{SPass}{writeCTM}.

The \Jlabel{spassWriteHypo}{-warpA} and \Jlabel{spassWriteHypo}{-rate}
options are important for Switchboard, I presume?

\subsection{\Jlabel{proc}{glatWriteHypo}}

Equivalent to \Jref{proc}{spassWriteHypo} for lattices. If you already
know the hypothesis, you can simply call this procedure with the
\Jlabel{spassWriteHypo}{-result} option, then it will not so a
\Jref{GLat}{rescore} to get it.

\subsection{\Jlabel{proc}{glatWriteTRN}}

Writes out a hypothesis in TRN format, otherwise equivalent to
\Jref{proc}{glatWriteHypo}. This format is produced by Lidia Mangu's
consensus code; this routine allows our results to be scored with the
same tools as hers.

\subsection{\Jlabel{proc}{metaLMloadWeights}}

Also available as \Jref{MetaLM}{loadWeights}, this procedure
reads in interpolation weights for a \Jref{module}{MetaLM} from a
standard weights file.

%%% Local Variables: 
%%% mode: latex
%%% TeX-master: t
%%% End: 
